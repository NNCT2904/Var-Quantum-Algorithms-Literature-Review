%  LaTeX support: latex@mdpi.com 
%  For support, please attach all files needed for compiling as well as the log file, and specify your operating system, LaTeX version, and LaTeX editor.

%=================================================================
\documentclass[journal,article,submit,pdftex,moreauthors]{Definitions/mdpi} 
% For posting an early version of this manuscript as a preprint, you may use "preprints" as the journal and change "submit" to "accept". The document class line would be, e.g., \documentclass[preprints,article,accept,moreauthors,pdftex]{mdpi}. This is especially recommended for submission to arXiv, where line numbers should be removed before posting. For preprints.org, the editorial staff will make this change immediately prior to posting.

%--------------------
% Class Options:
%--------------------
%----------
% journal
%----------
% Choose between the following MDPI journals:
% acoustics, actuators, addictions, admsci, adolescents, aerospace, agriculture, agriengineering, agronomy, ai, algorithms, allergies, alloys, analytica, animals, antibiotics, antibodies, antioxidants, applbiosci, appliedchem, appliedmath, applmech, applmicrobiol, applnano, applsci, aquacj, architecture, arts, asc, asi, astronomy, atmosphere, atoms, audiolres, automation, axioms, bacteria, batteries, bdcc, behavsci, beverages, biochem, bioengineering, biologics, biology, biomass, biomechanics, biomed, biomedicines, biomedinformatics, biomimetics, biomolecules, biophysica, biosensors, biotech, birds, bloods, blsf, brainsci, breath, buildings, businesses, cancers, carbon, cardiogenetics, catalysts, cells, ceramics, challenges, chemengineering, chemistry, chemosensors, chemproc, children, chips, cimb, civileng, cleantechnol, climate, clinpract, clockssleep, cmd, coasts, coatings, colloids, colorants, commodities, compounds, computation, computers, condensedmatter, conservation, constrmater, cosmetics, covid, crops, cryptography, crystals, csmf, ctn, curroncol, currophthalmol, cyber, dairy, data, dentistry, dermato, dermatopathology, designs, diabetology, diagnostics, dietetics, digital, disabilities, diseases, diversity, dna, drones, dynamics, earth, ebj, ecologies, econometrics, economies, education, ejihpe, electricity, electrochem, electronicmat, electronics, encyclopedia, endocrines, energies, eng, engproc, ent, entomology, entropy, environments, environsciproc, epidemiologia, epigenomes, est, fermentation, fibers, fintech, fire, fishes, fluids, foods, forecasting, forensicsci, forests, foundations, fractalfract, fuels, futureinternet, futureparasites, futurepharmacol, futurephys, futuretransp, galaxies, games, gases, gastroent, gastrointestdisord, gels, genealogy, genes, geographies, geohazards, geomatics, geosciences, geotechnics, geriatrics, hazardousmatters, healthcare, hearts, hemato, heritage, highthroughput, histories, horticulturae, humanities, humans, hydrobiology, hydrogen, hydrology, hygiene, idr, ijerph, ijfs, ijgi, ijms, ijns, ijtm, ijtpp, immuno, informatics, information, infrastructures, inorganics, insects, instruments, inventions, iot, j, jal, jcdd, jcm, jcp, jcs, jdb, jeta, jfb, jfmk, jimaging, jintelligence, jlpea, jmmp, jmp, jmse, jne, jnt, jof, joitmc, jor, journalmedia, jox, jpm, jrfm, jsan, jtaer, jzbg, kidney, kidneydial, knowledge, land, languages, laws, life, liquids, literature, livers, logics, logistics, lubricants, lymphatics, machines, macromol, magnetism, magnetochemistry, make, marinedrugs, materials, materproc, mathematics, mca, measurements, medicina, medicines, medsci, membranes, merits, metabolites, metals, meteorology, methane, metrology, micro, microarrays, microbiolres, micromachines, microorganisms, microplastics, minerals, mining, modelling, molbank, molecules, mps, msf, mti, muscles, nanoenergyadv, nanomanufacturing, nanomaterials, ncrna, network, neuroglia, neurolint, neurosci, nitrogen, notspecified, nri, nursrep, nutraceuticals, nutrients, obesities, oceans, ohbm, onco, oncopathology, optics, oral, organics, organoids, osteology, oxygen, parasites, parasitologia, particles, pathogens, pathophysiology, pediatrrep, pharmaceuticals, pharmaceutics, pharmacoepidemiology, pharmacy, philosophies, photochem, photonics, phycology, physchem, physics, physiologia, plants, plasma, pollutants, polymers, polysaccharides, poultry, powders, preprints, proceedings, processes, prosthesis, proteomes, psf, psych, psychiatryint, psychoactives, publications, quantumrep, quaternary, qubs, radiation, reactions, recycling, regeneration, religions, remotesensing, reports, reprodmed, resources, rheumato, risks, robotics, ruminants, safety, sci, scipharm, seeds, sensors, separations, sexes, signals, sinusitis, skins, smartcities, sna, societies, socsci, software, soilsystems, solar, solids, sports, standards, stats, stresses, surfaces, surgeries, suschem, sustainability, symmetry, synbio, systems, taxonomy, technologies, telecom, test, textiles, thalassrep, thermo, tomography, tourismhosp, toxics, toxins, transplantology, transportation, traumacare, traumas, tropicalmed, universe, urbansci, uro, vaccines, vehicles, venereology, vetsci, vibration, viruses, vision, waste, water, wem, wevj, wind, women, world, youth, zoonoticdis 

%---------
% article
%---------
% The default type of manuscript is "article", but can be replaced by: 
% abstract, addendum, article, book, bookreview, briefreport, casereport, comment, commentary, communication, conferenceproceedings, correction, conferencereport, entry, expressionofconcern, extendedabstract, datadescriptor, editorial, essay, erratum, hypothesis, interestingimage, obituary, opinion, projectreport, reply, retraction, review, perspective, protocol, shortnote, studyprotocol, systematicreview, supfile, technicalnote, viewpoint, guidelines, registeredreport, tutorial
% supfile = supplementary materials

%----------
% submit
%----------
% The class option "submit" will be changed to "accept" by the Editorial Office when the paper is accepted. This will only make changes to the frontpage (e.g., the logo of the journal will get visible), the headings, and the copyright information. Also, line numbering will be removed. Journal info and pagination for accepted papers will also be assigned by the Editorial Office.

%------------------
% moreauthors
%------------------
% If there is only one author the class option oneauthor should be used. Otherwise use the class option moreauthors.

%---------
% pdftex
%---------
% The option pdftex is for use with pdfLaTeX. If eps figures are used, remove the option pdftex and use LaTeX and dvi2pdf.

\usepackage{amssymb}
% \usepackage[leqno]{amsmath}
\usepackage{amsfonts}
\usepackage{amsopn}
\usepackage{amstext}
\usepackage{amsthm}
\usepackage[textsize=small]{todonotes}
%\usepackage[disable]{todonotes}
\setuptodonotes{color=blue!30}

% Packages for special symbols and ornaments:
\usepackage{calrsfs}
\usepackage{fourier-orns}
%\usepackage{hieroglf} \newcommand{\hiero}{\textpmhg}
\usepackage{clock} %\ClockFrametrue\ClockStyle2
\usepackage[alpine, weather]{ifsym}
\usepackage{tabularx} % extra features for tabular environment
\usepackage{graphicx} % takes care of graphic including machinery
\usepackage{blindtext}
\usepackage{soul}
\usepackage{physics}
\usepackage{qcircuit}
\usepackage{subcaption} 

\usetikzlibrary{cd}

%=================================================================
% MDPI internal commands
\firstpage{1} 
\makeatletter 
\setcounter{page}{\@firstpage} 
\makeatother
\pubvolume{1}
\issuenum{1}
\articlenumber{0}
\pubyear{2022}
\copyrightyear{2022}
%\externaleditor{Academic Editor: Firstname Lastname}
\datereceived{} 
%\daterevised{} % Only for the journal Acoustics
\dateaccepted{} 
\datepublished{} 
%\datecorrected{} % Corrected papers include a "Corrected: XXX" date in the original paper.
%\dateretracted{} % Corrected papers include a "Retracted: XXX" date in the original paper.
\hreflink{https://doi.org/} % If needed use \linebreak
%\doinum{}
%------------------------------------------------------------------
% The following line should be uncommented if the LaTeX file is uploaded to arXiv.org
%\pdfoutput=1

%=================================================================
% Add packages and commands here. The following packages are loaded in our class file: fontenc, inputenc, calc, indentfirst, fancyhdr, graphicx, epstopdf, lastpage, ifthen, lineno, float, amsmath, setspace, enumitem, mathpazo, booktabs, titlesec, etoolbox, tabto, xcolor, soul, multirow, microtype, tikz, totcount, changepage, attrib, upgreek, cleveref, amsthm, hyphenat, natbib, hyperref, footmisc, url, geometry, newfloat, caption

%=================================================================
%% Please use the following mathematics environments: Theorem, Lemma, Corollary, Proposition, Characterization, Property, Problem, Example, ExamplesandDefinitions, Hypothesis, Remark, Definition, Notation, Assumption
%% For proofs, please use the proof environment (the amsthm package is loaded by the MDPI class).

%=================================================================
% Full title of the paper (Capitalized)
\Title{Investigation of Barren Plateaus in\\ Quantum Neural Network Development}

% MDPI internal command: Title for citation in the left column
\TitleCitation{Title}

% Author Orchid ID: enter ID or remove command
\newcommand{\orcidauthorA}{0000-0000-0000-000X} % Add \orcidA{} behind the author's name
%\newcommand{\orcidauthorB}{0000-0000-0000-000X} % Add \orcidB{} behind the author's name

% Authors, for the paper (add full first names)
\Author{Thanh~Nguyen$^{1}$, Jacob Cybulski$^{1}$}

%\longauthorlist{yes}

% MDPI internal command: Authors, for metadata in PDF
\AuthorNames{Thanh Nguyen, Jacob Cybulski}

% MDPI internal command: Authors, for citation in the left column
\AuthorCitation{Nguyen, T., Jacob, C.}
% If this is a Chicago style journal: Lastname, Firstname, Firstname Lastname, and Firstname Lastname.

% Affiliations / Addresses (Add [1] after \address if there is only one affiliation.)
\address{%
$^{1}$ \quad Affiliation 1; ncng@deakin.edu.au
$^{2}$ \quad Affiliation 1; jacob.cybulski@deakin.edu.au
}

% Contact information of the corresponding author
\corres{Correspondence: ncng@deakin.edu.au}

% Current address and/or shared authorship
% \firstnote{Current address: Affiliation 3.} 
% \secondnote{These authors contributed equally to this work.}
% The commands \thirdnote{} till \eighthnote{} are available for further notes

%\simplesumm{} % Simple summary

%\conference{} % An extended version of a conference paper

% Abstract (Do not insert blank lines, i.e. \\) 
\abstract{\emph{Quantum neural networks} (QNN) are machine learning models which aim to utilise quantum computing algorithms to improve training regimes for classical artificial neural networks.
One of the approaches to implementing QNNs is to employ \emph{variational quantum algorithms} (VQA), which take advantage of optimisation algorithms executed on classical hardware to train parametrised quantum circuits, while using quantum machines to model the landscape of the loss function and efficiently estimate its gradient.
As compared with classical neural networks, using VQA allows training of a well-designed QNN to rapidly converge to a solution, while avoiding many problems commonly present in training classical neural networks.
QNNs however have their own trainability issues, which can manifest in large variational QNN circuits, either due to their depth, the large number of qubits, or poor initialisation of circuit parameters. One such problem is the emergence of large flat areas in the loss function landscape, called \emph{barren plateaus}, which impede effective circuit optimisation.
The methods of preventing the formation of barren plateaus or mitigating their presence in circuit optimisation have been proposed and their variants are vigorously pursued. However, efficacy of each method in the context of the QNN architecture and training data remains an open question.
This project therefore aims to investigate the effectiveness of different approaches to dealing with barren plateaus in various QNN developmental circumstances.
The project is undertaken in two phases.
In the first phase, three different approaches are selected to deal with barren plateaus and are then evaluated against different VQA quantum circuit structures (depth, qubits and initialisation) and a random gradient landscape.
In the second phase, performance of the selected methods is analysed and compared when used with several QNNs built using standard data sets.}

% Keywords
\keyword{Quantum Computing, Quantum Machine Learning, Variational Quantum Algorithms, Barren Plateaus, Python Programming, Qiskit} 

% The fields PACS, MSC, and JEL may be left empty or commented out if not applicable
%\PACS{J0101}
%\MSC{}
%\JEL{}

%%%%%%%%%%%%%%%%%%%%%%%%%%%%%%%%%%%%%%%%%%
% Only for the journal Diversity
%\LSID{\url{http://}}

%%%%%%%%%%%%%%%%%%%%%%%%%%%%%%%%%%%%%%%%%%
% Only for the journal Applied Sciences
%\featuredapplication{Authors are encouraged to provide a concise description of the specific application or a potential application of the work. This section is not mandatory.}
%%%%%%%%%%%%%%%%%%%%%%%%%%%%%%%%%%%%%%%%%%

%%%%%%%%%%%%%%%%%%%%%%%%%%%%%%%%%%%%%%%%%%
% Only for the journal Data
%\dataset{DOI number or link to the deposited data set if the data set is published separately. If the data set shall be published as a supplement to this paper, this field will be filled by the journal editors. In this case, please submit the data set as a supplement.}
%\datasetlicense{License under which the data set is made available (CC0, CC-BY, CC-BY-SA, CC-BY-NC, etc.)}

%%%%%%%%%%%%%%%%%%%%%%%%%%%%%%%%%%%%%%%%%%
% Only for the journal Toxins
%\keycontribution{The breakthroughs or highlights of the manuscript. Authors can write one or two sentences to describe the most important part of the paper.}

%%%%%%%%%%%%%%%%%%%%%%%%%%%%%%%%%%%%%%%%%%
% Only for the journal Encyclopedia
%\encyclopediadef{For entry manuscripts only: please provide a brief overview of the entry title instead of an abstract.}

%%%%%%%%%%%%%%%%%%%%%%%%%%%%%%%%%%%%%%%%%%
% Only for the journal Advances in Respiratory Medicine
%\addhighlights{yes}
%\renewcommand{\addhighlights}{%

%\noindent This is an obligatory section in “Advances in Respiratory Medicine”, whose goal is to increase the discoverability and readability of the article via search engines and other scholars. Highlights should not be a copy of the abstract, but a simple text allowing the reader to quickly and simplified find out what the article is about and what can be cited from it. Each of these parts should be devoted up to 2~bullet points.\vspace{3pt}\\
%\textbf{What are the main findings?}
% \begin{itemize}[labelsep=2.5mm,topsep=-3pt]
% \item First bullet.
% \item Second bullet.
% \end{itemize}\vspace{3pt}
%\textbf{What is the implication of the main finding?}
% \begin{itemize}[labelsep=2.5mm,topsep=-3pt]
% \item First bullet.
% \item Second bullet.
% \end{itemize}
%}

%%%%%%%%%%%%%%%%%%%%%%%%%%%%%%%%%%%%%%%%%%
\begin{document}

%%%%%%%%%%%%%%%%%%%%%%%%%%%%%%%%%%%%%%%%%%

% \section{Project Description}

\subsection{Background}\label{Background Section}
\emph{Quantum computing} is a relatively new field of Science, which adopts and merges elements of Physics, Mathematics, and Computer Science.
Algorithms for quantum computers have been developed to deal with problems that belong to the non-deterministic polynomial time complexity class, and thus are challenging to solve using classical computers \cite{williamsSolvingNPCompleteProblems2011,jiangQuantumAnnealingPrime2018,farhiQuantumApproximateOptimization2014}.
Such quantum algorithms are commonly represented in a form of \emph{quantum circuits} consisting of \emph{qubits} (or quantum bits) and \emph{quantum gates} (fundamental quantum operations). A programming language such as Python is often used to construct quantum circuits, which could subsequently be executed in a simulator or using the actual quantum hardware.


Unfortunately, today's quantum computers are still in an early stage of technological development, they are severely resource constrained and are vulnerable to the environmental noise, and so at this stage their applications are rather limited.
They are all described as the Noisy Intermediate-Scale Quantum (NISQ) \cite{brooksQuantumSupremacyHunt2019} computers, which from the viewpoint of their technological maturity can be compared to the first programmable computers a century ago.
Virtually all NISQ computers face the same gate control precision and data execution issues: the absence of fault-tolerant design to counter errors due to qubit decoherence (breakdown of their quantum state), the limited number of qubits available to a quantum processor, and restrictions imposed on the depth (or size) of quantum circuits that can be reliably processed and executed in a quantum machine.
Moreover, quantum circuits, which encode both quantum algorithms and their input data, are static in their design.
Consequently, every new input data to a quantum algorithm necessitates construction of a new and different quantum circuit.
Thus, unlike classical algorithms, quantum algorithms (as encoded in quantum circuits) cannot be directly parameterised or reused. This implies that quantum machine learning models, which are learning models relying on quantum algorithms, must rely on unique approaches to their training.

A common approach to addressing the above-mentioned constraints is to adopt a hybrid approach to developing quantum machine learning, i.e. by combining quantum circuits and their classical optimization.
Variational Quantum Algorithms (VQAs) \cite{cerezo2021variational} can produce quantum circuit templates that can be instantiated with trainable parameters, which are then reusable for quantum computer execution.
At the same time, the classical optimiser sees the variational circuits as a black box that returns results from inputs and the trainable parameter.

The hybrid quantum-classical method is also the idea behind Quantum Neural Networks (QNNs) \cite{altaisky2001quantum}.
Different constructing methods lead to different definitions of QNNs \cite{paetznick2013, zhaoBuildingQuantumNeural2019, caoQuantumNeuronElementary2017}.
However, they share the same criteria as pointed out by Schuld, Sinayskiy, and Petruccione \cite{schuldQuestQuantumNeural2014}:
(1) The initial state can encode any binary string;
(2) The calculation process reflects Neural Networks principles;
(3) The system's evolution is based on and entirely consistent with quantum theory.
\almarginpar{The entire description of QNN is very imprecise and confusing \\ - I have changed this paragraph a bit, is it more clear?}
At the current stage, QNN is seen as a subclass of VQA, consisting of variational circuits and classical optimizers \cite{abbasPowerQuantumNeural2021}.
The variational method has enabled many classical machine learning algorithms to implement their alternatives for quantum computers \cite{hugginsQuantumMachineLearning2019, takakiLearningTemporalData2021, shinguBoltzmannMachineLearning2021, dallaire-demersQuantumGenerativeAdversarial2018}.
In this project, we implement a multi-layered Quantum Perceptron \cite{kristensenArtificialSpikingQuantum2021} as the parameterized circuit for our experiment activities.

% Some known types of QNN for the recent quantum processor are: 
% Quantum Tensor Neural Network (QTNN) \cite{hugginsQuantumMachineLearning2019} which achieved a balance of computational efficiency and expressive power. 
% The tensor network can reduce the required qubits to process high-dimensional data with powerful optimization algorithms.
% Quantum Recurrent Neural Network (QRNN) is constructed as a parameterized circuit \cite{takakiLearningTemporalData2021}, with some qubits being initialized and measured at each step while others memorize the past data.
% However, whether this quantum alternative is better than the classical Recurrent Neural Network is still an open question.
% The NISQ processors are also capable of delivering some other QNN models such as: 
% Quantum Boltzmann Machine \cite{shinguBoltzmannMachineLearning2021}\cite{zoufalVariationalQuantumBoltzmann2021}, 
% Quantum Generative Adversarial Network \cite{dallaire-demersQuantumGenerativeAdversarial2018}\cite{lloydQuantumGenerativeAdversarial2018}.
% Studies have shown that QNN performance and trainability can be significantly higher compared to its classical counterpart on today's hardware \cite{abbasPowerQuantumNeural2021, colesSeekingQuantumAdvantage2021}, and has several applications, for example, breast cancer prediction \cite{liModelAlgorithmQuantuminspired2014}, or image processing \cite{matsuiQubitNeuralNetwork2009}. 

As VQA is the mainstream method for designing QNN circuits, QNN inevitably inherited some shortcomings from VQA.
One of which is the training difficulties, known as \emph{Barren Plateaus} (BP).
This phenomenon happens when training a QNN framework with a comparatively large number of qubits; the objective function becomes flat and leads to difficulties in finding the loss function minimum using the gradient descent method \cite{mccleanBarrenPlateausQuantum2018, zhaoAnalyzingBarrenPlateau2021}, thus causing inefficiency in circuit training.
Figure \ref{fig: Barren Plateau Example} provides an example illustration of BP.
This problem was pointed out  by Abbas et al \cite{abbasPowerQuantumNeural2021}.
However, the author left this problem for further study.
The BP of QNN design under VQA is therefore worth investigating.

\begin{figure}[h]
    \centering
    \includegraphics[width=\textwidth]{src/Appendices/example-of-a-barren-plateau.png}
    \caption{
        An illustrative example of a barren plateau and narrow gorge.
        On both plots: the expectation value of a Hamiltonian for a single parameter in the quantum circuit.
        On the left: expectation value landscape in the absence of barren plateau.
        On the right expectation value landscape in case of a barren plateau.
        Figure from Chen et al. \cite{tillyVariationalQuantumEigensolver2021}.
    }
    \label{fig: Barren Plateau Example}
\end{figure}

This research project undertakes a study that aims to survey and compare some countermeasure approaches to mitigate or avoid the effect of BP in the QNN development under VQA methods.
The main work of this article is summarized:
We provide some essential background for VQA and QNN;
We define the BP phenomenon and the causes that would lead to this issue;
The composition of known methods to address the matter of concern will be introduced, as well as performance comparison to identify the best approach;
Finally, we conclude the paper with some open issues and prospects for the field.


\subsection{The problem} 
\label{Problem Section}
For this research project, we are interested in the research question: "How to mitigate or avoid Barren Plateaus (BP) in the QNN development in the VQA circuit construction approach." The context is given as in section \ref{Background Section}.

\subsection{Objectives}
To answer the above research question, it will be necessary to meet the following research objectives:
% \begin{enumerate}
%     \item Provide the background information for quantum computing, VQA and QNN, which enables definition of the research question (completed, see \ref{Background Section});
%     \item Define the BP problem and the causes that would lead to this phenomenon \cite{wangNoiseinducedBarrenPlateaus2021,zhaoAnalyzingBarrenPlateau2021} (completed, see \ref{Problem Section});
%     \item Investigate several methods to mitigate or avoid BP \cite{pesahAbsenceBarrenPlateaus2021, pattiEntanglementDevisedBarren2021,liuParameterInitializationMethod2021};
%     \item Compare advantages and disadvantages of the identified methods to identify the most appropriate approaches in different circumstances;
%     \item Conclude the research with summary and open issues still remaining for the future research.
% \end{enumerate}
\begin{itemize}
    \item Trimester 1 - 2022: Literature review and research design
    \begin{enumerate}
        \item Provide the background information for quantum computing, VQA and QNN, which enables definition of the research question (completed, see \ref{Background Section});
        \item Define the BP problem and the causes that would lead to this phenomenon \cite{wangNoiseinducedBarrenPlateaus2021,zhaoAnalyzingBarrenPlateau2021} (completed, see \ref{Problem Section});
        \item Investigate several methods to mitigate or avoid BP \cite{pesahAbsenceBarrenPlateaus2021, pattiEntanglementDevisedBarren2021,liuParameterInitializationMethod2021} (completed as minimum viable artefact, see \ref{Minimum Artefacts});
    \end{enumerate}
    \item Trimester 2 - 2022: Addressing the gaps
    \begin{enumerate}
        \item Investigate several methods to mitigate or avoid BP \cite{pesahAbsenceBarrenPlateaus2021, pattiEntanglementDevisedBarren2021,liuParameterInitializationMethod2021};
    \item Compare advantages and disadvantages of the identified methods to identify the most appropriate approaches in different circumstances;
    \item Conclude the research with summary and open issues still remaining for the future research.
    \end{enumerate}
\end{itemize}

\subsection{Success Criteria}
\begin{enumerate}
    \item Review 3-4 different methods of dealing with BP
    \item Determine which method is the most appropriate in what context;
    \item Provide guidelines for mitigating BP;
\end{enumerate}



\subsection{Motivation}
There are indeed some problems with QNN training, as pointed out in section \ref{Background Section}, namely the Barren Plateaus phenomenon.
Luckily, some studies addressed the problem: \cite{pesahAbsenceBarrenPlateaus2021,pattiEntanglementDevisedBarren2021,liuParameterInitializationMethod2021}.S
We bring together those methods to identify the most suitable for different circumstances.
We also give a reference and guidelines for mitigating or avoiding BP, and therefore, VQAs can solve more practical problems. 
This project provides an approach to Quantum Algorithms for Machine Learning and Computer scientists.
% \begin{titlepage}
    % \begin{center}
    %     \vspace*{1cm}
            
    %     \Huge
    %     \textbf{Exploration of Variational Quantum Algorithm}
            
    %     \vspace{0.5cm}
    %     \LARGE
    %     \title{I will pick a title later}
        
            
    %     \vspace{1.5cm}
            
    %     \textbf{Thanh Nguyen and Jacob Cybulski}
            
    %     \vfill
            
    %     A thesis presented for the degree of\\
    %     Bachelor of IT Honours
            
    %     \vspace{0.8cm}
            
    %     \includegraphics[width=0.4\textwidth]{src/CoverPage/DeakinUniversityLogo.jpg}
        
            
    %     \Large
    %     % Faculty of Science, Engineering and Built Environment\\
    %     Deakin University\\
    %     \date{\today}
            
    % \end{center}
% \end{titlepage}  
\thispagestyle{empty}
\begin{titlepage}
    \includegraphics[width=0.25\textwidth]{src/CoverPage/Deakin_Logo.jpeg}
        \begin{center}
        \vspace*{4cm}
        \todo{May change later on}
        {\LARGE Investigation of Barren Plateaus in Quantum Neural Network development} %%Replace this with the Title of your research
        \vspace{3cm}
            \begin{large}   
    
        
            \bf Submitted as Honours Dissertation in SIT723/SIT724
            \vspace{1cm}
        
            \bf \today \\
            T1-2022        
        
            \vspace{3cm}
            \textbf{Thanh Nguyen}\\
            STUDENT ID 218583133 \\
            COURSE - Bachelor of IT Honours (S470)
            \vfill

            \bf \normalsize Supervised by: Prof, Jacob Cybulski\\
       
        \end{large}  
   \end{center}
\end{titlepage}


\section{Draft Structure}
Send to Jacob by 24 Mar, submit the draft by 27 Mar.

Some references are available at \href{https://www.youtube.com/watch?v=0ENQXz9tDww&list=PLOFEBzvs-VvqJwybFxkTiDzhf5E11p8BI&index=18}{Qiskit Global Summer School 2021}: 
\cite{mccleanBarrenPlateausQuantum2018, cerezoCostFunctionDependent2021, romeroQuantumAutoencodersEfficient2017, beerTrainingDeepQuantum2020, sharmaTrainabilityDissipativePerceptronbased2020, grantHierarchicalQuantumClassifiers2018, congQuantumConvolutionalNeural2019, pesahAbsenceBarrenPlateaus2021, wangNoiseinducedBarrenPlateaus2021, skolikLayerwiseLearningQuantum2021, grantInitializationStrategyAddressing2019,holmesConnectingAnsatzExpressibility2022,volkoffLargeGradientsCorrelation2021, marreroEntanglementinducedBarrenPlateaus2021, pattiEntanglementDevisedBarren2021, verdonLearningLearnQuantum2019, carolanVariationalQuantumUnsampling2020, manginiQuantumComputingModels2021}

A basic classification of the literature artifacts
\begin{itemize}
    \item Take one step back to review QNN, VQA, and their problems: \cite{mccleanBarrenPlateausQuantum2018, cerezo2021variational};
    \item Think like a table, identify the gaps of articles, topic by topic, then what solution each article offers: \cite{skolikLayerwiseLearningQuantum2021, grantInitializationStrategyAddressing2019, cerezoCostFunctionDependent2021}
    \item Synthesize in a compact way;
\end{itemize}

\input{LiteratureReview/Content/classification.tex}

\bibliographystyle{jcabbrv} % Sorted and "note" fields removed
\bibliography{src/References/zoteroReferences.bib}
% % \begin{titlepage}
    % \begin{center}
    %     \vspace*{1cm}
            
    %     \Huge
    %     \textbf{Exploration of Variational Quantum Algorithm}
            
    %     \vspace{0.5cm}
    %     \LARGE
    %     \title{I will pick a title later}
        
            
    %     \vspace{1.5cm}
            
    %     \textbf{Thanh Nguyen and Jacob Cybulski}
            
    %     \vfill
            
    %     A thesis presented for the degree of\\
    %     Bachelor of IT Honours
            
    %     \vspace{0.8cm}
            
    %     \includegraphics[width=0.4\textwidth]{src/CoverPage/DeakinUniversityLogo.jpg}
        
            
    %     \Large
    %     % Faculty of Science, Engineering and Built Environment\\
    %     Deakin University\\
    %     \date{\today}
            
    % \end{center}
% \end{titlepage}  
\thispagestyle{empty}
\begin{titlepage}
    \includegraphics[width=0.25\textwidth]{src/CoverPage/Deakin_Logo.jpeg}
        \begin{center}
        \vspace*{4cm}
        \todo{May change later on}
        {\LARGE Investigation of Barren Plateaus in Quantum Neural Network development} %%Replace this with the Title of your research
        \vspace{3cm}
            \begin{large}   
    
        
            \bf Submitted as Honours Dissertation in SIT723/SIT724
            \vspace{1cm}
        
            \bf \today \\
            T1-2022        
        
            \vspace{3cm}
            \textbf{Thanh Nguyen}\\
            STUDENT ID 218583133 \\
            COURSE - Bachelor of IT Honours (S470)
            \vfill

            \bf \normalsize Supervised by: Prof, Jacob Cybulski\\
       
        \end{large}  
   \end{center}
\end{titlepage}


\section{Research Design tasks}
\begin{enumerate}
    \item A Project Description with the aims and scope defined
    \item List of research questions to be answers, and motivation
    \item List of resources may required to address the research questions, how to access those resources.
    \item Methodologies and experiment designs
    \item Overview of data required and how to collect data.
\end{enumerate}


\section{Research Question and Motivation}
For this research project, we are interested in the problem "How to mitigate or avoid Barren Plateaus (BP) in the QNN development in the VQA circuit construction approach."

There are indeed some problems with QNN training, as pointed out in Project Description and Literature review, namely the Barren Plateaus phenomenon.
Luckily, some studies have addressed the problem: \cite{pesahAbsenceBarrenPlateaus2021,pattiEntanglementDevisedBarren2021,liuParameterInitializationMethod2021}.
We bring together those methods to identify the most suitable for different circumstances.
We also give a reference and guidelines for mitigating or avoiding BP, and therefore, VQAs can solve more practical problems. 


% \section{Resources}

Most of our resources are open-sources

\begin{itemize}
    \item Python 3.6+: \url{https://www.python.org/downloads/}
    \item Anaconda: \url{https://www.anaconda.com/products/distribution}
    \item Jupyter Notebook: \url{https://jupyter.org/}
    \item Qiskit: \url{https://qiskit.org/}
\end{itemize}

Setup guidelines for Qiskit: 
\url{https://qiskit.org/documentation/getting_started.html}

\section{Methodologies and Experiments}
\todo{do as suggested!}
\begin{figure}
    \includegraphics[width=\textwidth]{./ResearchDesign/Appendices/VarianceShrinking.png}
    \caption{
        An example of Barren Plateaus phenomenon occurs to a QNN model. 
        The variance of the gradient shrinking \textit{exponentially} with the number of qubits. 
        Barren Plateaus phenomenon prevents optimization algorithms to navigate the cost function landscape efficiently.
    }
    \label{Variance Shrinking demo}
\end{figure}

Cerezo et al. has pointed out that Barren Plateaus is cost function dependent \cite{cerezoCostFunctionDependent2021}, which implies that the only way to fully eliminate Barren Plateaus is to use a local cost function with a shallow circuit.
However, there are still other approaches to mitigate the phenomenon without using the local cost function.
The papers \cite{skolikLayerwiseLearningQuantum2021, liuParameterInitializationMethod2021} suggests that we can initiate the starting parameters away from plateaus to guarantee the trainability from the first steps.

First, we create a QNN model to reproduce the Barren Plateaus. 
We can use Qiskit to gain access to a wide range of ansatz, optimization algorithms, and most importantly the quantum emulator that capable of simulate up to 32 qubits.
The literature review also suggested that the two factors causing Barren Plateaus are the \textit{ansatz depth} and the \textit{randomised starting parameters}.
To reproduce the Barren Plateaus we force the QNN model to have one of the two factors, or both at the same time.
Then we calculate the variance of the gradient, and notice if the variance is shrinking \textit{exponentially} with the number of qubits (see Figure \ref{Variance Shrinking demo}).

\todo{I need to work on this section more, the methods sound trivial}

Then, we implement the three methods as discussed in the literature review.
The three methods will be applied to each QNN model (deep layered ansatz, randomised starting parameters, or both).
We will track the variances of the gradient for each QNN model and method.

After practising in a simulator, we experiment on the quantum hardware

From the data collected in experiment, we can conclude which method is the most suitable for different situation.



% \section{Data Required}
\begin{itemize}
    \item A QNN model based on VQA, that possess Barren Plateaus
\end{itemize}

\bibliographystyle{jcabbrv} % Sorted and "note" fields removed
\bibliography{src/References/zoteroReferences.bib}
% \section{Artefact Development Report - T1 2022 (draft)}
\almarginpar{We need some introduction to the section 4 explaining its purpose and its}
\almarginpar{Be clear from above design discussion what the artefact development was to achieve and what part of it the minimum artefact satisfies}
\begin{todolist}
    \item[\done] Discussion on artefact developed (so far) and evaluation of the artefact (if any);
    \item[\done] Details of data collected, experimental procedure;
    \item Lessons learned in the sprint (?);
    \item[\done] Results of the artefact analysis;
\end{todolist}

Notes from Prof.

\begin{todolist}
    \item[\done] How gradient variance is being calculated (formula)
    \item[\done] How do we know that high variance does not simply mean that your gradient function is not just noise (for future works)?
    \item Does it converge eventually (for future works)?
    \item Can you verify that the gradient descent results in a solution (for future works)?
\end{todolist}

In this section, we describe the progress on the artefact development, the detail and context for the experiments are given as Section \ref{Research Design section}.
For T1 - 2022, we partly address the phase 1 of the process (see Section \ref{Research Activities section}) by implement a minimum viable artefact that contains a series of exploration experiments.
The minimum viable artefact is a Python notebook file containing Python scripts to run the experiment.
We run the notebook with IBM Quantum Experience, as they provide online services for simulating quantum hardware.

This minimum artefact will cover the ansatz component of the QNN and VQA model.
We have chosen two ansatzes \emph{NLocal} and \emph{TwoLocal} as the objects of study.
The two treatments applied to these ansatzes are the unrestricted ansatz (method \#0) and local cost function - shallow circuit (method \#1) (see Table \ref{implementation of methods table} and Figure \ref{Research Activities Figure}).
We summarise the experiment results from Section \ref{Result section} as the table \ref{Experiment summary table}.

\begin{table}
    \centering
    \begin{tabular}{|| c c c ||}
        \hline
        Ansatz   & Method                             & Variance of gradients \\[0.5ex]
        \hline \hline
        NLocal   & No restriction                     & Decay                 \\
        \hline
        TwoLocal & No restriction                     & Decay                 \\
        \hline
        NLocal   & Local cost function, shallow depth & Sustain               \\
        \hline
        TwoLocal & Local cost function, shallow depth & Sustain               \\
        \hline
    \end{tabular}
    \caption{
        The experiments that we implemented in the Python Notebook. We test the same ansatzes with different methods, we record the results as the vanishing rate of gradient when the number of qubits increased.
    }
    \label{Experiment summary table}
\end{table}

\subsection{The Quantum Provider}
For this experiment, we are using the quantum emulator provided by Qiskit.
The QASM simulator is used to mimic an IBMQ device.
Additionally, the QASM simulator, by default, has no noise, so we can expect the result to be noise-free.
Note that the fault-free emulators do not reflect quantum devices precisely as the actual devices may suffer from various types of noise.
However, considering the allowed time span for these experiments, we will use the QASM simulator for T1-2022 and leave the actual quantum devices for future works.

\subsection{Creating Ansatzes}
We have chosen the \textit{NLocal} and \textit{TwoLocal} classes from the Qiskit circuit library to explore the two ansatzes structures

An example of circuits generated by Qiskit is visualised in Figure \ref{Ansatz samples}.
We can generate different ansatz by altering the repetition number and qubit number.
The circuit depth is the largest number of gate operations across all qubit registers in a circuit.
Furthermore, as the circuit high-level definition is translated into the gate set available on a given quantum machine, the circuit depth may significantly increase.
Obviously, as the ansatz repetition grows, the circuit depth also grows.
Figure \ref{Ansatz samples} further shows that the higher number of qubits leads to a deeper circuit for a fully entangled ansatz.

\begin{figure}
    \includegraphics[width=\textwidth]{Artefact/Appendices/ansatz3-2.png}
    \includegraphics[width=\textwidth]{Artefact/Appendices/ansatz3-3.png}
    \includegraphics[width=\textwidth]{Artefact/Appendices/ansatz4-2.png}
    \caption{
        Samples of parameterised circuits generated by the Qiskit framework with 'full entanglement' option.
        The ansatz is a sequence of rotation layers and entanglement layers.
        Above: an ansatz of three qubits and two repetition layers.
        Middle: an ansatz of three qubits and three repetition layers.
        Below: an ansatz of four qubits and two repetition layers.
    }
    \label{Ansatz samples}
\end{figure}

\subsection{Treatments for ansatzes}
Here we implement the two treatments for the ansatzes.
We select the first two methods from Section \ref{Research Design section} and applied them to the two selected ansatz.

\subsubsection{Method \#0: Unrestricted}
As discussed in the Section \ref{Research Design section}, the goal of this configuration is to produce a general multilayer perceptron network without any restriction.
The ansatzes will have unrestricted growth of circuit depth with a global cost function, the initial parameters of this ansatz is randomised.
We implement the default ansatz to have the number of qubits and repetition increased iteratively.

\subsubsection{Method \#1: Local Cost Function and Shallow Depth Implementation}
We implemented the \textit{Global Cost Function} as the measurement output for all qubits, while the \textit{Local Cost Function} is the measurement for the first two qubits.
Section \ref{Shallow Circuits, Local Cost Function section} and figure \ref{cost functions} previously explained the differences between the two cost functions.
The shallow ansatz is the same as compared with the default, however, the repetition number is kept as a constant number.





\subsection{Visualise the Variance}
To calculate the gradient variance, we use the parameter shift rule from Eq. (\ref{Parameter-shift rules}) as implemented in Qiskit Gradient Framework.
The BP phenomena can be verified when the gradient variance decreases with an increased number of qubits and repetition layers.

To visualise the gradient variances, we have plotted a range of random parameters for each ansatz as the initial starting point.
Such randomised parameters are generated 100 times uniformly to calculate the gradients.
We then plot the variance values of the gradients for different numbers of qubits and repetition values for a range of 2 to 9 qubits and ansatz layer repetition.
Note that the neural network generated in this experiment is not designed to answer a problem, we will focus on the trainability of each method in later steps of the experiment.

In short, we use 100 uniformly randomised parameters to scan the gradient, then we calculate the "slope" of the gradient.

\subsection{Results and Analysis}

\begin{todolist}
\item mention a table that includes the number of qubits/layer/random parameter for each ansatz
\item describe more in figure \ref{Variance Local Cost}
\item analysis instead of describe
\item add the figures above (python notebook) for analysis
\item describe how alternating data leads to different result(s).
\end{todolist}

The section \ref{Development Process section} dicusses the configurations of the experiment. 
We summarize the experiment results as the table:
\begin{center}
    \begin{tabular}{|| c c c ||}
        \hline
        Ansatz      & Method                                & Variance of gradients \\[0.5ex] 
        \hline \hline
        NLocal      & None                                  & Decay                 \\
        \hline
        TwoLocal    & None                                  & Decay                 \\
        \hline
        NLocal      & Local Cost Function, Shallow circuit  & Sustain               \\
        \hline
        TwoLocal    & Local Cost Function, Shallow circuit  & Sustain               \\
        \hline
    \end{tabular}
\end{center}

For the default setting, the two ansatzes' gradient variances decay as expected.
As the number of qubits scales up, the variances decay exponentially, this indecates that the cost function landscape becomes flatter and flatter. 
We suspect that the result would be inefficientcy for any gradient-based optimization algorithm to train the model.
We discussed this phenomenon in Section \ref{Barren Plateaus section}.
Figure \ref{Plot ansatzes gradients default} shows the results of the two ansatzes in this configuration.

\begin{figure}
    \includegraphics[width=\textwidth]{Artefact/Appendices/NLocalDefault.png}
    \centerline{a) NLocal Ansatz gradient variance values}
    \includegraphics[width=\textwidth]{Artefact/Appendices/TwoLocalDefault.png}
    \centerline{b) TwoLocal Ansatz gradient variance values}
    \caption{
        The variances of gradient from differences ansatzes.
        On both plots: the variances vanish exponentially to the number of qubits, ansatzes are in default configuration.
    }
    \label{Plot ansatzes gradients default}
\end{figure}

In contrast, for the case of Local Cost Function and Shallow circuit, we observe that the variances of the ansatzes' gradient did not vanished when we attempt to increase the number of qubits.
This implies that the cost function landscape can sustain the slope.
Figure \ref{Variance Local Cost} shows the result of the experiment for Local Cost Function and Shallow circuit, in comparison with the default settings.

\begin{figure}
    \includegraphics[width=\textwidth]{Artefact/Appendices/variancesLCF.png}
    \caption{
        Comparison of the variance values of the two ansatzes with and without Local Cost Function and constant depth.
        The ansatzes with Global Cost Function and increased depth have their gradient variances decay exponentially with the number of qubits. 
    }
    \label{Variance Local Cost}
\end{figure}


\subsection{Artefact Development Summary}

We have implemented three methods of dealing with barren plateau by altering the ansatz's depth, cost function, and initial parameters aspects.
The experiments have produced the results as the slopes of the gradients for a number of qubits, as well as the performance of ansatzes in neural network training.
The results indicate that the variances of the gradient can be stable if we set a limit on the length of the circuit and the cost function, and the training performance can increase if we carefully select the initial parameters.

With this artefact, we have addressed the research design in Section \ref{Research Design section}.
We have implemented the three methods, namely \textit{local cost function, shallow ansatz depth}, \textit{layerwise learning} and \textit{identity blocks}.
We have compared the variances in the first phase of the experiment.
We anticipate that a higher variance value does not mean the optimisation is going in the right direction, the model can still stick it in a local minimum, or the error landscape is random.
To verify the performance of these methods, in the second phase, we implemented a variational quantum neural network to solve a classification problem with a standard dataset.

The experiments have verified that the unrestricted configuration may have run into a barren plateau, while the others can converge to their minimums - the answers.
Moreover, these experiments are conducted in an emulated environment with a noise model from \emph{ibm\_perth} backend.
Thus, this experiment can reflect the real-life situation to some extent.

According to Figure \ref{Fig: Plot Variances}, the variance values of limited depth - cost function method at seven qubits configuration is significantly higher than the rest.
At higher qubit configuration, the barren plateaus are also more likely to appear, and the performance of these methods under this phenomenon is worth investigating.
Due to time constrain, we leave these experiments for future works, including ansatzes of higher qubit count, and a dataset of higher dimension for classification.


% \section{Conclusion}
We have seen the effect of gradient concentrates to zero, namely \emph{barren plateaus}.
The causes of this phenomenon are related to the depth of the circuit, the width of the qubit registry to measure, and the randomly initialised parameters.
This vanishing gradients phenomenon makes it difficult for the gradient-based optimisation algorithms to find the optimal solution (i.e. the global minimum).

There are multiple techniques to mitigate the issue \cite{cerezoCostFunctionDependent2021, skolikLayerwiseLearningQuantum2021, grantInitializationStrategyAddressing2019}.
One approach is to address the structure of the parameterised circuit by utilising a local cost function and limitation in circuit depth.
Another possibility is to pre-train or generate the initial parameters such that the starting point would not land on a plateau.
We designed a series of exploratory experiments with QNN models to answer the research question.

In machine learning, the loss function which guide the optimisation process, would reflects the internal performance in dealing with barren plateaus, while the accuracy of the model describes the practical performance.
The loss function, in combination with the structure of neural network would translate to the trainability of the model.
This experiment has demonstrated the internal performance of quantum variational classifier ansatzes in the phase one and the practical performance as accuracy score in phase two.

% We have delivered a comparison of methods to construct the ansatz component of the QNN model.
We compared their effectiveness and performances by analysing their gradient decay rates and their accuracy by solving a classification problem.
The results indicate that unrestricted configuration leads to exponential decay in the gradient variance for each qubit added to the ansatzes.
On the other hand, the local cost function, combined with restricted depth, can sustain ansatzes gradient variance for higher qubit count, resulting in better accuracy in the classification test.
For the ansatzes with lower depth, we can expect more learning capacity.
The initial parameter factor in theses ansatzes would significantly contribute to the accuracy in the classification problem, with some trade-offs between time and quality.

Our study thus provide guidedance for further implementations of ansatzes for hybrid quantum algorithms in experiment.
We leave the relationship between training data, capacity and barren plateaus for future works, as well as further experiments with more qubits, and gradient descent with hyperparameter tuning.



%%%%%%%%%%%%%%%%%%%%%%%%%%%%%%%%%%%%%%%%%%
\begin{adjustwidth}{-\extralength}{0cm}
    %\printendnotes[custom] % Un-comment to print a list of endnotes

    \reftitle{References}

    % Please provide either the correct journal abbreviation (e.g. according to the “List of Title Word Abbreviations” http://www.issn.org/services/online-services/access-to-the-ltwa/) or the full name of the journal.
    % Citations and References in Supplementary files are permitted provided that they also appear in the reference list here. 

    %=====================================
    % References, variant A: external bibliography
    %=====================================
    %\bibliography{your_external_BibTeX_file}

    %=====================================
    % References, variant B: internal bibliography
    %=====================================
    % \begin{thebibliography}{999}

    % \end{thebibliography}
    \bibliography{src/References/zoteroReferences.bib}
    % If authors have biography, please use the format below
    %\section*{Short Biography of Authors}
    %\bio
    %{\raisebox{-0.35cm}{\includegraphics[width=3.5cm,height=5.3cm,clip,keepaspectratio]{Definitions/author1.pdf}}}
    %{\textbf{Firstname Lastname} Biography of first author}
    %
    %\bio
    %{\raisebox{-0.35cm}{\includegraphics[width=3.5cm,height=5.3cm,clip,keepaspectratio]{Definitions/author2.jpg}}}
    %{\textbf{Firstname Lastname} Biography of second author}

    % For the MDPI journals use author-date citation, please follow the formatting guidelines on http://www.mdpi.com/authors/references
    % To cite two works by the same author: \citeauthor{ref-journal-1a} (\citeyear{ref-journal-1a}, \citeyear{ref-journal-1b}). This produces: Whittaker (1967, 1975)
    % To cite two works by the same author with specific pages: \citeauthor{ref-journal-3a} (\citeyear{ref-journal-3a}, p. 328; \citeyear{ref-journal-3b}, p.475). This produces: Wong (1999, p. 328; 2000, p. 475)

    %%%%%%%%%%%%%%%%%%%%%%%%%%%%%%%%%%%%%%%%%%
    %% for journal Sci
    %\reviewreports{\\
    %Reviewer 1 comments and authors’ response\\
    %Reviewer 2 comments and authors’ response\\
    %Reviewer 3 comments and authors’ response
    %}
    %%%%%%%%%%%%%%%%%%%%%%%%%%%%%%%%%%%%%%%%%%
\end{adjustwidth}
\end{document}

