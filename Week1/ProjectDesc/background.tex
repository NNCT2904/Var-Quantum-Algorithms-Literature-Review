\subsection{Background}

\todo{you really need to fix your English throughout}Regardless of \todo[prepend]{vast?} far-fetched power of supercomputers, some problems \todo{just "remain"}remain unsolved, as their complexity scales too quickly even for some top-of-the-line computing systems to handle. 
We need to research and develop new tools with enough processing power to process those tasks, which can take the forms of molecular behaviours analysing in chemistry, factoring large integers for cryptography, or machine learning.

Quantum computers were developed based on Quantum mechanics to deliver a more excellent processing scaling than classical ones.
It is the relic of the collaboration between modern physics, engineering, and computer science.
To demonstrate the advantages of Quantum computers, Google claimed "Quantum supremacy" in 2019 by solving a problem that takes about 10000 years for supercomputers in 200 seconds \cite{hsuGoogleQuantumTech2019}.
New records for this Quantum race are established repeatedly, such as Jiuzhang Quantum computer from China with 56 Qbits configuration, reported having 10 billion times in performance compared to Google's \cite{zhongQuantumComputationalAdvantage2020}, or the most recent 127 Qbits system "Eagle" from IBM \cite{chow2021ibm}.

Both Classical and Quantum computers suffer the same fundamental issue of data corruption.
For Classical computers, the cause of this fault would be magnetism, electricity and radiation.
However, there are even more things to be concerned about in the case of Quantum computers, not only from magnetism or electricity and radiation. They require separation from heat, light, and vibration. In other words, absolute isolation from the outside world.

Such close systems are, of course, only exist in fiction. 
To some extent, all Quantum computers are designed to be controlled by interaction with the outside world, or we cannot input or retrieve data.
It also means that some disturbances that we have no control of can interrupt the system or interact with the outside world.
These interruptions, called "decoherence", are the primary factor that destroys the Quantum evolution needed for large scale algorithms on Quantum computers.

A way to counter decoherence is using Quantum Error Correction codes \cite{lidar2013quantum} to protect quantum information, by using multiple physical Qbits to represent a logical Qbit, such as the 9-Qbit Shor code \cite{shor1995scheme}.
Considering the high cost of fault-tolerant Quantum computers and their maintenance, Qbits must be used as efficiently as possible. 
The limited amount of Qbits per processor and circuit depth due to noise at this time also narrow down on the scale of algorithms that Quantum computers can execute.

Nowadays, to develop a quantum algorithm, high-level programming languages such as Python are used to generate quantum circuits, which can be executed by a simulator or by a quantum computer. This leads to another issue: the initial state of Qubits in registries are often being prepared statically as either value $\ket{0}$ or $\ket{1}$

The availability of those fault-tolerant Quantum computers is not likely to happen soon. 
While waiting for those devices, scientists had discovered a new approach for the current early and noisy Qbit registers.
Variational Quantum algorithms (VQAs) \cite{cerezo2021variational} attempted to address these constraints by using machine learning methods to train a parameterised quantum circuit.
\todo{the preceding discussion does not address the issue of Variational Quantum Algorithms, the fact that QNN is an example of VQA, the reason for investigating these, etc. }
\begin{center}
    \vspace{10pt}
    Add more later
    \vspace{10pt} 
\end{center}
Perhaps, something like this should be in this section: You need more focus, do the following:
\begin{enumerate}
    \item open with a general statement about quantum computing and their benefits as compared with classical computers, 
    \item explain how quantum algorithms are being developed and the fact that they are "static",
    \item discuss that an alternative involved hybrid solutions, such as VQA and give some examples of VQAs, including QNNs, 
    \item briefly discuss/mention different types of QNNs, their advantages and disadvantages, 
    \item set a claim that this project undertakes a study aims to compare different approaches to implementing QNNs and contrasting them with classical NNs, 
    \item finally mention that the research will focus on a classification problem (e.g. of images).
\end{enumerate}
