\begin{abstract}
\emph{Quantum neural networks} (QNN) are machine learning models which aim to utilise quantum computing algorithms to improve training regimes for classical artificial neural networks.
One of the approaches to implementing QNNs is to employ \emph{variational quantum algorithms} (VQA), which take advantage of optimisation algorithms executed on classical hardware to train parametrised quantum circuits, while using quantum machines to model the landscape of the loss function and efficiently estimate its gradient. 
As compared with classical neural networks, using VQA allows training of a well-designed QNN to rapidly converge to a solution, while avoiding many problems commonly present in training classical neural networks. 
QNNs however have their own trainability issues, which can manifest in large variational QNN circuits, either due to their depth, the large number of qubits, or poor initialisation of circuit parameters. One such problem is the emergence of large flat areas in the loss function landscape, called \emph{barren plateaus}, which impede effective circuit optimisation. 
The methods of preventing the formation of barren plateaus or mitigating their presence in circuit optimisation have been proposed and their variants are vigorously pursued. However, efficacy of each method in the context of the QNN architecture and training data remains an open question. 
This project therefore aims to investigate the effectiveness of different approaches to dealing with barren plateaus in various QNN developmental circumstances. 
The project is undertaken in two phases. 
In the first phase, three different approaches are selected to deal with barren plateaus and are then evaluated against different VQA quantum circuit structures (depth, qubits and initialisation) and a random gradient landscape.
In the second phase, performance of the selected methods is analysed and compared when used with several QNNs built using standard data sets.
\\\\
{\bf Keywords:} Quantum Computing, Quantum Machine Learning, Variational Quantum Algorithms, Barren Plateaus, Python Programming, Qiskit.

\end{abstract}