\begin{abstract}
Quantum Computing is a new and exciting area of Science that intersects Physics, Mathematics and Computer Science.
Quantum algorithms are known to solve problems, which until recently have been considered unsolvable classically due to their inherent complexity. 
Variational Quantum Algorithm (VQA) is a hybrid classical-quantum technique that can produce variational (or parametrised) circuits, which act as templates for circuit generation. 
VQA adopted the machine learning technique such that the quantum hardware is used to estimate the cost function gradient while using classical optimisation algorithms to train the quantum circuit.
VQA also have some trainability issues, namely Barren Plateaus.
This phenomenon occurs when we train a circuit with a large number of qubits.
While Barren Plateaus remains an open question, there are attempts to mitigate the effect of this issue.

This project aims to investigate a range of approaches to avoid or eliminate the effect of Barren Plateaus in QNN and VQA development.
Our work in the project is summarised: 
We provide some background on VQA, QNN and their trainability issue; 
We implement several methods to address the Barren Plateaus with Python and Qiskit framework;
We compare the trainability of the model by solving a problem with QNN.
\\\\
{\bf Keywords:} Quantum Computing, Quantum Machine Learning, Variational Quantum Algorithms, Python Programming, Qiskit.

\end{abstract}