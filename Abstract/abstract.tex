\begin{abstract}
Quantum Computing is a new and exciting area of Science, which intersects Physics, Mathematics and Computer Science. Quantum algorithms are known to solve problems, which until recently have been considered unsolvable classically due to their inherent complexity. Typically a high-level quantum algorithm in Python generates a quantum circuit, which can be executed on a real quantum machine or a simulator. Unfortunately, quantum circuits hard-code their data, so that an application of an algorithm to new data leads to a new circuit. This means that circuits are static and cannot be trained or optimised directly. However, there exist hybrid classical-quantum techniques that can produce variational (or parametrised) circuits, which act as templates for circuit generation. Quantum machine learning algorithms take advantage of variational circuits to implement quantum-alternatives to many machine learning algorithms. In this way, quantum solutions can be both highly efficient and data rich.

This project aims to investigate a range of variational quantum algorithms, i.e. hybrid classical-quantum algorithms capable of manipulating parametrised quantum circuits.
\\\\
{\bf Keywords:} Quantum Computing, Quantum Machine Learning, Variational Quantum Algorithms, Python Programming, Qiskit.

\end{abstract}