\subsection{Resources} \label{Resources section}
Most of our required resources are open-sources:
\begin{itemize}
    \item Python 3.6+: \url{https://www.python.org/downloads/}
    \item Jupyter Notebook: \url{https://jupyter.org/}
    \item Qiskit: \url{https://qiskit.org/}
    \item IBM Quantum: \url{https://quantum-computing.ibm.com/}
    \item Qiskit circuit library: \url{https://qiskit.org/documentation/apidoc/circuit_library.html}
\end{itemize}

To prepare the quantum emulator on a local machine, we first install Python and Anaconda for programming language support and Jupyter Notebook as a code editing tool.
Then we follow the official instruction from Qiskit \cite{Qiskit} to install the necessary packages.
The quantum emulator from Qiskit is capable of simulating up to 32 qubits.
We can start working with Qiskit in a Jupyter Notebook file.

\almarginpar{Python "kernel"? do you mean "API" \\ - How about "python environment"? A notebook kernel is a “computational engine” that executes the code contained in a Notebook document.}The other option is to use the provided Python notebook environment provided by IBM Quantum with Qiskit pre-installed.
While this is a convenient choice for online presentation or remote working because of instant access, this server has shortcomings.
The maximum ram for open access is only 8 Gigabytes, and the processing power is limited.
The actual quantum hardware from IBM is limited to 5 qubits for open access.

\almarginpar{It also offers custom ansatz design, would you explore this option, why not?}Qiskit offers various pre-defined ansatz designs, namely \textit{EfficientSU2, PauliTwoDesign, RealAmplitudes, NLocal}, and \textit{TwoLocal}.
The Qiskit circuit library also allows custom parameterized circuits to be used as ansatzes.
Each of the circuit templates is used for different purposes, such as quantum chemistry, quantum neural network, quantum eigensolver, etc.
They also have their advantages or disadvantages.
For example, it is common to use the \textit{RealAmplitudes} ansatz to implement a multi-layer perceptron neural network.