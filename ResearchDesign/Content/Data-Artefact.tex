\subsection{Data Collecting Method and Artefacts Development}
\label{Data Collecting Section}
Here we discuss our method of collecting data from a series of experiments.
We implement the artefact as a Python Notebook that includes all experiment activities and results.
The notebook will require some pre-installed resources to execute (refer to Section \ref{Resources section} for further detail).
As we are conducting experiments on quantum circuits, quantum hardware is also required.
The notebook will be configured to communicate with an IBM quantum device.

% I tried to describe the output form of the python notebook, but it is more like a duplicate with section Research Activities above, so I removed this section
% \almarginpar{I do not understand this}After the first phase, we will obtain the selected ansatzes objects with the four treatments applied (see Figure \ref{Research Activities Figure}) and their variance shrinking rates.
% The shrinking rate can be recorded as a list of numbers and presented as line graphs (for example, see Figure \ref{Variance Shrinking demo}).
% With this data, we can fulfil the first criteria in the Section \ref{Criteria section} by comparing the shrinking rate of each method.

% For the second phase, we will need to further develop the ansatzes from phase one into QNN models. We use the same input data and the optimiser across all models to preserve clarity.
% We run the models along with a benchmark script that records the progress and properties of models (refer to Section \ref{Criteria section})

% Finally, we summarise the research results and present the findings.

\subsubsection{Success Condition}
The experiment will be concluded when these conditions are met:
\begin{itemize}
    \item We have ansatzes that can produce barren plateaus;
    \item The methods are implemented as Python scripts and ready for demonstration;
    \item The ansatzes are implemented as QNN models to solve problems.
    \item The outcomes are evaluated against the criteria.
\end{itemize}