\subsection{Data Collecting Method and Artefacts Development}
\label{Data Collecting Section}
Here we discuss our method to collect the data from the series of experiments.
We implement the artefact as a Python Notebook, all experiment activities and results will be included within the artefect.
The notebook will require some pre-installed resources in order to execute (refer to Section \ref{Resources section} for further detail).
As we are conducting experiment on quantum circuits, a quantum hardware is also required.
The notebook will be configured to communicate with a IBM quantum device.

After the first phase, we will obtain the ansatzes object respectively to the treatments as described in Section \ref{Research Activities section} and their variance shrinking rates.
The shringking rate can be recorded as a list of number, and can be presented as line graphs (for example see Figure \ref{Variance Shrinking demo}).
With this data, we can fulfill the first criteria in Section \ref{Criteria section} by compare the shrinking rate of each method.

For the second phase, we will need to further develope the ansates from phase one into QNN models, we use the same the input data and the optimizer across all models to preserve clarity.
We run the models along with a benchmark script that record the progress and properties of models (refer to Section \ref{Criteria section})

Finally, we summarise the research results and present the findings.

\subsubsection{Success Condition}
The experiment will be concluded when these conditions are met:
\begin{itemize}
    \item We have ansatzes that can produce barren plateaus;
    \item The methods are implemented as Python scripts and ready for demonstration;
    \item The ansatzes are implemented as QNN models to solve problems.
    \item The outcomes are evaluated against the criteria.
\end{itemize}