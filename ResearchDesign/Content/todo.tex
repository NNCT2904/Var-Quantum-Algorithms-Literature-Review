\section{Research Design tasks}
\begin{todolist}
    \item[\done] A Project Description with the aims and scope defined
    \item[\done] List of research questions to be answers, and motivation
    \item List of resources may require to address the research questions, how to access those resources.
    \item Methodologies and experiment designs
    \item Overview of data required and how to collect data.
\end{todolist}

Top down and Bottom up research:
\begin{itemize}
    \item Inductive research: Bottom up research, we start with some observation of patterns, then propose a hypothesis for a new theory. Inductive research is about finding new theories from existing data, hence "Bottom up";
    \item Deductive research: Top down research, we start with some theory, then propose a hypothesis, then observation to confirm the hypothesis.
\end{itemize}

Qualitative and Quantitative research
\begin{itemize}
    \item Qualitative research: statistical analysis on number and data, population and sample sizes.
    \item Quantitative research: the quality of data. People opinion and understanding and motivation, immeasurable by number
\end{itemize}

Tell what methods are being used, and justify them.
Why these methodologies are the best for the research?

\subsection{Notes from Prof.}
It would be great seeing the completed research design as per previous discussion.
Here is a recap of what research design chapter should include: 
\begin{todolist}
    \item[\done] Research question and objectives restated, 
    \item[\done] The overall research methodology / approach suitable the research associated with the research question - in this context it is experiments, 
    \item[\done] Research methods to be used in the project (as implied and be consistent with the methodology), \textbf{done (?)}
    \item[\done] Research plan as consisting of a schedule of research activities, such as calibration of instruments, experiments and their objectives, evaluation / validation of experimental results, etc., 
    \item[\done] Research resources needed for the tasks, such as data, software, programs / algorithms, computer resources, etc., 
    \item[\done] Conditions for research success, e.g. accuracy to be achieved, results to be obtained, any external verification of the methods to be deployed, and validation of results to be applied and determined (how would this be done).
\end{todolist}