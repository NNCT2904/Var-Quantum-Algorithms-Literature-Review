\section{Minimum Artefact and Future Works}

The success conditions of the experiment, as discussed, contains the QNN models, the three implementations, the data generated and a guideline to identify which methods to be used on different occasion.
However, the time for the experiment of the unit is limited to one week. We believe that we can deliver a minimum viable artefact as a single python notebook that contains:
\begin{itemize}
    \item QNN ansatzes that possesses Barren Plateaus;
    \item Implementation of three methods as discussed in the literature review;
    \item Verify the existence of Barren Plateaus for each method above by calculating the variances of their gradients.
\end{itemize}

This artefact outcome will not answer the research question completely. 
However, it covers the first half of the experiment and ensures that the outcome is ready to advance to the next steps.
We leave the second half of the experiment for future works.
In more detail, we record the data produced from the three methods in the criteria in Section \ref{Criteria section}. 
Then we perform comparisons and synthesise the data to verify or reject the hypotheses.
