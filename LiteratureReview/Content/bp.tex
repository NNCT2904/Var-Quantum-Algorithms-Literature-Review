\subsection{Barren Plateaus, Gradients, Trainability Issue} \label{Barren Plateaus section}

When training a quantum circuit, initially there are three important elements: some reference states (the inputs), a parameterised unitary operation (the circuit, or ansatz), and an observable $O$ to construct a cost function.
The next step is to use this parameterised circuit with some classical optimisation algorithm on classical hardware.
Each run of the optimisation routine receives the current value of the cost function and suggests the new set of parameters for the circuit \cite{cerezo2021variational}.
By changing the parameter, we are exploring an optimisation landscape to find a global minimum.
The properties of this landscape heavily depend on the structure of the parameterised circuits and the cost function.

Typically, two starting ingredients for problem-solving using heuristic ansatzes are a randomly parameterized circuit $U$ and a set of random initialization parameters $\vec{\theta}$, non-parameterized unit $W_l$ for each layer $l$ \cite{mccleanBarrenPlateausQuantum2018}:
\begin{equation}\label{Parameterized Circuit}
    U(\vec{\theta})
    = U(\theta_1, \cdots, \theta_L)
    = \prod_{l=1}^L U_l(\theta_l)W_l
\end{equation}
With $L$ is the depth of the circuit. Consider a cost function $C(\theta)$ with an observable $O$ and the ansatz $U$:
\begin{equation}
    C(\vec{\theta})
    = \bra{0} U(\vec{\theta})^\dagger OU(\vec{\theta}) \ket{0}
\end{equation}

The gradients are the derivatives of the cost function respectively to the parameters and have a severe impact on the performance of the training model:
\begin{equation}
    \partial_l C = \frac{\partial C(\vec{\theta})}{\partial\theta_l}
\end{equation}
McClean et al. \cite{mccleanBarrenPlateausQuantum2018} also pointed out that whenever the random parameterised circuits reach the depth $O(n^{1/L})$ on a $L$-dimensional array, the gradient of the cost function will vanish, and the variance of the gradient shrinks (see Eq. (\ref{Vanish Gradient})) exponentially with the number of qubits (see Eq. (\ref{Variance expo smaller})).
\begin{align}
    \langle \partial_k C\rangle &= 0  \label{Vanish Gradient}\\
    \mathrm{Var}[\partial_k C] &\approx 2^{-n}  \label{Variance expo smaller}
\end{align}
Barren Plateaus is a serious issue for circuit optimisation because the flatter the landscape is, the more challenging it is to search for the location of the global cost minimum.
Compared to the gradient in a deep classical network which vanishes exponentially in the number of layers, for the QNN case, the gradient is exponentially small in the number of qubits \cite{mccleanBarrenPlateausQuantum2018}.

Additionally, the choice of the initiate parameter $\theta$ is also an important factor. When an ansatz is randomly initialised, the algorithm may start distant from the solution, at a local minimum, or even on a barren plateau area. 

To summarise, there are factors that lead to Barren Plateaus in QNN and VQA:
\begin{itemize}
    \item \textbf{The ansatz depth and number of qubit;}
    \item \textbf{Random initialization parameters;}
\end{itemize}

