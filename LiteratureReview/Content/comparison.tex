\section{Comparison and Disscussion}
\begin{itemize}
    \item make a table;
    \item Characteristics, pros and cons;
    \item Highlight which occasion for which method, context of QNN
    \item Comparison, compact way
\end{itemize}

We now discuss the characteristics, pros and cons of each methods.
Let's recall the main cause of the Barren Plateaus phenomenon in VQA and QNN development are the ansatz depth and the randomised starting parameter.
For this reason, we will identify the characteristics of each method based on how they address these two factors.

The paper \cite{cerezoCostFunctionDependent2021} by Cerezo et al. address the issue by propose their own Alternating Layered Ansatz with a defined upper-bound and lower-bound for circuit depth to work with their local cost function. 
The initialise parameters can be randomly configured.
Thus, this methods can be use to train a shallow circuit of length in the defined bound. 
Overall, this is an ansatz design that does not exhibit a Barren Plateaus.

Grant et al. \cite{grantInitializationStrategyAddressing2019} choose to mitigate the issue by addressing the initialisation parameters. In short, the authors' idea was to divide the ansatz into shallow blocks