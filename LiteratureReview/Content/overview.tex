\subsection{Overview}
This section reviews the QNNs and VQAs, their structures, and the mathematics underneath. 
We aim to explore their gaps in studies, especially the factors that would lead to the problem of Barren Plateaus.
Then, we review and compare some methods to counter or minimise the Barren Plateaus phenomenon by addressing those factors.

In general, Barren Plateaus can be noise-induced \cite{wangNoiseinducedBarrenPlateaus2021}, which means that the noise from quantum hardware affects the trainability; 
or circuit-induced \cite{mccleanBarrenPlateausQuantum2018}, as a result of the circuit design and initialisation parameters.
This research focuses on mitigating circuit-induced Barren Plateaus by reviewing some of the available studies \cite{pesahAbsenceBarrenPlateaus2021, cerezoCostFunctionDependent2021, skolikLayerwiseLearningQuantum2021}.
From this point onward, we will address the term 'Circuit-Induced Barren Plateaus', which will be referred to as 'Barren plateaus' for simplicity.

We strongly advise readers to have a background in quantum computing, linear algebra, and machine learning. 
The 2020 Qiskit Global Summer School course \cite{2020QiskitGlobal} is a good start for beginners; the course provides some basic mathematics, theories and practices. 
The book by Sutor \cite{sutorDancingQubitsHow2019} is also highly recommended. 
Other than the basics of quantum Computing, the author presented an overview of the current quantum hardware and simulators.
Finally, the 2021 Qiskit Global Summer School course \cite{2021QiskitGlobal} was focused on the selected aspects of machine learning for quantum devices.