\subsection{Overview}
This section reviews the QNNs and VQAs, their structures, and their underpinning mathematics. 
We aim to explore the gaps in the previous studies, \almarginpar{Not sure if "the gaps" were addressed in this thesis version, perhaps to be completed in T2 \\ - Do I need to specify that? after all, we will handover the current document to T2 \\ - It may need to be stated somewhere, perhaps in the first section? \\ - done sir, section 1.3}specifically the factors that would lead to the problem of barren plateaus.
Then, we review and compare some methods of countering or minimising the barren plateaus phenomenon by addressing those factors.

In general, barren plateaus can either be noise-induced \cite{wangNoiseinducedBarrenPlateaus2021}, which means that the noise from quantum hardware affects the trainability; 
or circuit-induced \cite{mccleanBarrenPlateausQuantum2018}, which is the result of a circuit design and its initialisation parameters.
This research focuses on mitigating circuit-induced barren plateaus by reviewing some of the prior studies of this phenomenon \cite{pesahAbsenceBarrenPlateaus2021, cerezoCostFunctionDependent2021, skolikLayerwiseLearningQuantum2021}.
From this point on, we will refer to the notion of 'Circuit-Induced Barren Plateaus' as 'barren plateaus' for simplicity.

In our discussion we will assume that the readers have some background knowledge of quantum computing, linear algebra, and machine learning. 
To gain such a background knowledge, we recommend the 2020 Qiskit Global Summer School course \cite{2020QiskitGlobal}, which provides some basic mathematics, theories and practices of quantum computing. 
The book by Sutor \cite{sutorDancingQubitsHow2019} also provides the foundations of quantum Computing plus an overview of issues related to the design of current quantum hardware and simulators.
Finally, the 2021 Qiskit Global Summer School course \cite{2021QiskitGlobal} was focused on the selected aspects of machine learning for quantum devices.