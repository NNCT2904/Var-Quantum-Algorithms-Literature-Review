\section{Overview}
This paper reviews the Quantum Neural Networks and Variational Quantum Algorithms, their structures, and the mathematics underneath. 
We aim to explore their gaps in studies, especially the factors that would lead to the problem of Barren Plateaus.
Then, we review and compare some methods to counter minimise the Barren Plateaus phenomenon by addressing those factors.

In general, Barren Plateaus can be noise-induced \cite{wangNoiseinducedBarrenPlateaus2021}, which means that the noise from quantum hardware affects the trainability; 
or circuit-induced \cite{mccleanBarrenPlateausQuantum2018}, as a result of the circuit design and initialisation parameters.
This research focuses on mitigating the effect of circuit induced Barren Plateaus by reviewing some of the available studies \cite{pesahAbsenceBarrenPlateaus2021, cerezoCostFunctionDependent2021, skolikLayerwiseLearningQuantum2021}.
From this point onward, we will address the term 'Noise-Induced Barren Plateaus' as 'Barren plateaus' for simplicity.

We strongly recommend readers to have foundation knowledge in Quantum Computing, Linear Algebra and Machine Learning. 
The 2020 Qiskit course \cite{2020QiskitGlobal} is a good start for beginners, this course provide very basic mathematics, theories and practices. 
The book \cite{sutorDancingQubitsHow2019} by Robert S. is also recommended while taking the 2020 Qiskit course, other than the basics of Quantum Computing, the author presented an overview of the current quantum hardware and simulator