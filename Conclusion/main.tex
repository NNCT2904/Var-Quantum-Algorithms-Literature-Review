\section{Conclusion}
We have seen the effect of gradient concentrates to zero, namely \emph{barren plateaus}.
The causes of this phenomenon are related to the depth of the circuit, the width of the qubit registry to measure, and the randomly initialised parameters.
This vanishing gradients phenomenon makes it difficult for the gradient-based optimisation algorithms to find the optimal solution (i.e. the global minimum).

There are multiple techniques to mitigate the issue \cite{cerezoCostFunctionDependent2021, skolikLayerwiseLearningQuantum2021, grantInitializationStrategyAddressing2019}.
One approach is to address the structure of the parameterised circuit by utilising a local cost function and limitation in circuit depth.
Another possibility is to pre-train or generate the initial parameters such that the starting point would not land on a plateau.
We designed a series of exploratory experiments with QNN models to answer the research question.

We have delivered a comparison of methods to construct the ansatz component of the QNN model.
We compared their effectiveness and performances by analysing their gradient decay rates and solving a classification problem.
The results indicate that unrestricted configuration leads to exponential decay in the gradient variance for each qubit added to the ansatzes.
On the other hand, the local cost function, combined with restricted depth, can sustain ansatzes gradient variance for higher qubit count, resulting in better accuracy in the classification test.
Furthermore, the initial parameter factor significantly improved the models' accuracy in the classification problem, with some tradeoffs between time and quality.

This experiment has demonstrated the performance of quantum variational classifier ansatzes under two qubits configuration.
In future works, we need to investigate the ansatzes of higher qubit count, and datasets with more features.

% Therefore the trainability of gradient-based algorithm on the two ansatzes would be better compared to the ones without any restriction.
% Compared to the expected artefact, we are still missing the last two methods, a problem so that the QNN models can be compared with each other, and most importantly, the results from the actual quantum devices  for further detail.
% We will cover these issues in further studies.