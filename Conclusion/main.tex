\section{Conclusion T1-2021}
We have seen the effect of gradient concentrates to zero, namely \emph{barren plateaus}.
The causes of this phenomena are related to the depth of circuit, the width of qubit registry to measure, the and randomly initialized parameters.
This vanishing gradients phenomena makes difficulties for the gradient-based optimization algorithms to find the optimal solution (i.e. the global minimum).

There are multiple techniques to mitigate the issue \cite{cerezoCostFunctionDependent2021,skolikLayerwiseLearningQuantum2021,grantInitializationStrategyAddressing2019}.
One appoach is to address the structure of the parameterized circuit, by utilising a local cost function and limiation in circuit depth.
Another posibility is to pre-train the initial parameters such that the the starting point would not land into a plateau.
We design a series of exploratory experiment with QNN models in order to answer the research question.

In the allowed time spand of T1-2021, we have delivered a \emph{minimum viabile artefact} to construct the ansatz component of the QNN model.
We also implement two methods as reviewed and compare their effectiveness by analysing their gradient decay rates.
Compared to the expected artefact, we are still missing the last two methods, a problem so that the QNN models can be compared with each others, and most importantly, the results from the actual quantum devices  for further detail.
We will cover these issues in further studies.