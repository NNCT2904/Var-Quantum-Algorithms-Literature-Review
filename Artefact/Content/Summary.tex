\subsection{Artefact Development Summary T1-2021}
We have implemented one method of dealing with barren plateau by altering the ansatz's depth and cost function aspects.
Due to the time constrain, these experiments are within the scope of a \textit{minimum viable artefact}.
The experiments have produced the results as the slopes of the gradients for a range of qubits.
The results indicate that the variances of the gradient can be stable if we set a limit on the length of the circuit and the measurement for qubits.

With this artefact, we have partly addressed the research design in Section \ref{Research Design section}, by exploring different types of ansatzes and comparing their gradient decay rate with two methods.
This artefact contributes the ansatz component and implementation of two methods to the final artefact.

Compared to the final artefact described in Section \ref{Data Collecting Section}, we expect to implement the rest of the three methods, namely \textit{layerwise learning} and \textit{identity blocks}.
Moreover, one shortcoming of the current artefact version is that we have not yet implemented a problem for the QNN model to optimise.
We have known that the method in the notebook can produce high variance values. 
However, it does not mean the optimisation is going in the right direction, the model can still stick it in a local minimum, or the error landscape is random. 
Thus, the experiments cannot verify if the model cannot converge to the global minimum - the answer.
Moreover, these experiments are conducted in an emulated environment without noise, decoherence, or faulty gates.
The actual quantum device will not have such characteristics.
Thus, this experiment is not reflecting the real-life situation.