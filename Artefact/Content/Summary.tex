\subsection{Artefact Development Summary T1-2021}
We have implemented one method of dealing with barren plateau by altering the depth and the cost function aspects of the ansatz.
Due to the time constrain, this experiment is only a \textit{minimum viable artefact}.
The experiments have produced the result as the slope of the gradients for a range of qubits.
The results indicate that the variance of the gradient can be stable if we set a limit on the length of circuit, as well as the measurement for qubits.
With this artefact, we have partly addressed the research question in Section \ref{Problem Section}, by comparing the the gradient decay rate of the ansatzes before and after applying the custom local cost function and restricting the depth.

Compared to the expected artefact, we are still missing the last two methods, a problem so that the QNN models can be compared with each others, and most importantly the results from the actual quantum devices.