\subsection{Treatments for ansatzes}
Here we implement the two treatments for the ansatzes.
We selected the first two methods from Section \ref{Research Design section} and applied them to the two selected ansatz.

\subsubsection{Method \#0: Unrestricted}
As discussed in the Section \ref{Research Design section}, the goal of this configuration is to produce a general multilayer perceptron network without any restriction.
The ansatzes will have unrestricted growth of circuit depth with a global cost function. 
The initial parameters of this ansatz are randomised.
We implement the default ansatz to have the number of qubits and repetition increased iteratively.

\subsubsection{Method \#1: Local Cost Function and Shallow Depth Implementation}
We implemented the \textit{Global Cost Function} as the measurement output for all qubits, while the \textit{Local Cost Function} is the measurement for the first two qubits.
Section \ref{Shallow Circuits, Local Cost Function section} and figure \ref{cost functions} previously explained the differences between the two cost functions.
The shallow ansatz is the same as the default. 
However, the repetition number is kept as a constant number.