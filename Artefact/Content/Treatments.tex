\subsection{Treatments for ansatzes}
Here we implement the two treatments for the ansatzes.
We select the first two methods from Section \ref{Research Design section} and applied them to the two selected ansatz.

\subsubsection{Method \#0: Unrestricted}
As discussed in the Research Design section \ref{Research Design section}, we configure the ansatz objects such that initially their gradient variances decrease exponentially with the number of qubits.
These characteristics are:
\begin{itemize}
    \item The circuit depth;
    \item The number of qubits to be measured for the cost function;
    \item The randomised parameters.
\end{itemize}
The ansatzes with default configuration will have unrestricted growth of circuit depth.
We implement the default ansatz to have the number of qubits and repetition increased iteratively.

\subsubsection{Method \#1: Local Cost Function and Shallow Depth Implementation}
We implemented the \textit{Global Cost Function} as the measurement output for all qubits, while the \textit{Local Cost Function} is the measurement for the first two qubits.
Section \ref{Shallow Circuits, Local Cost Function section} and figure \ref{cost functions} previously explained the differences between the two cost functions.
The shallow ansatz is the same as compared with the default, however, the repetition number is kept as a constant number.