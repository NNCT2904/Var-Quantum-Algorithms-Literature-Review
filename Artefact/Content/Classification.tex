\subsection{Classification Problem} \label{Sec: Classification Problem}
We implement a variational quantum classifier as previously discussed in the section \ref{VQA}.
We summarise the algorithm as the circuit in Figure \ref{Fig: Quantum circuit for classifier}.
The algorithm includes two stages: a training stage and a classification stage.
We use a dataset generated by Qiskit machine learning package for the algorithm.

The dataset consited of labeled datapoints with two features is used for the traning stage.
The quantum circuit will be sampled as required by Gradient Descent algorithm provided by Qiskit to minimise the cost function and thus optimise the ansatz.

For the classification satge, we use another set of datapoints without labels, and run the classifier with the optimised parameters obtained from the training stage.
The classifier would produce the predicted labels for each testing datapoint, and we compare these predicted label to the actual provided labels.
Finally, we obtain the success ratio of the classifier.

The quantum circuit for both stages is constructed from three parts: the feature map to encode data, the ansatz to perform optimisation and finally, a final rotation layer for measurement (the cost function).



\begin{figure}
    \centerline{
    \Qcircuit @C=1em @R=2em {
    \lstick{\ket{0}} & \multigate{1}{U_{\phi}(\vec{x})}    & \multigate{1}{W(\vec{\theta})}    & \meter & \rstick{z_1} \cw \\
    \lstick{\ket{0}} & \ghost{U_{\phi}(\vec{x})}           & \ghost{W(\vec{\theta})}           & \meter & \rstick{z_n} \cw \\
    }
    }
    \caption{
        Quantum Variational Classifier implemented for the experiment.
        The initial qubit state $\ket{0}^n$ is applied with a feature map $U_{\phi}(\vec{x})$ to encode data $\vec{x}$ from our dataset.
        Then, an unitary operation $W(\vec{\theta})$ is applied as the ansatz, follow up with a measurement layer.
        The output string $z \in \{0,1\}^n$ is mapped to a label in the dataset.
    }
    \label{Fig: Quantum circuit for classifier}
\end{figure}