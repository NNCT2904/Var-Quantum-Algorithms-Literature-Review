\section{Artefact Development Report}
\label{Minimum Artefact section}
\todo{Change this section to include T2 works}
In this section, we describe the progress on the artefact development, the detail and context for the experiments are given as Section \ref{Research Design section}.
We address the process (see Section \ref{Research Activities section}) by implementing a series of exploratory experiments.
The minimum viable artefact is a Python notebook file containing Python scripts to run the experiment.
We run the notebook with IBM Quantum Experience, as it provides online services for running and simulating quantum hardware.
We use the hardware noise sample from the device \emph{ibm\_perth} to configure our simulator to achieve a most accurate to that of a real quantum neural network training scenario.

We have chosen the ansatz \emph{Real Amplitude} and a custom ansatz (see Section \ref{Sec: Creating Ansatzes}) as the objects of study.
The four treatments applied to these ansatzes are further discussed in Section \ref{Sec: Treatments for ansatzes} (see Table \ref{implementation of methods table} and Figure \ref{Research Activities Figure}).
\almarginpar{Need to update this table, or make another one to include the classification problem}
We summarise the experiment results of the first phase from Section \ref{Result section} in Table \ref{Experiment summary table}.

\begin{table}
    \centering
    \begin{tabular}{|| l l p{2.2cm} p{1.7cm} ||}
        \hline
        \textbf{Ansatz} & \textbf{Method}                         & \textbf{Variance exponential fit} & \textbf{Accuracy score} \\[0.5ex]
        \hline \hline
        Real Amplitude  & \#0: No restriction                     & -0.63                             & 72.5\%                  \\
        \hline
        Real Amplitude  & \#1: Local Cost function, shallow depth & -0.06                             & 80.0\%                  \\
        \hline
        Real Amplitude  & \#2: Layerwise Learning                 & -0.57                             & 92.5\%                  \\
        \hline
        Customised      & \#3: Identity Blocks                    & -0.60                             & 90.0\%                  \\
        \hline
    \end{tabular}
    \caption{
        The experiments that we implemented in the Python Notebook. We test the same ansatzes with different methods, we record the results as the vanishing rate of gradient when the number of qubits increased.
        The differences between the \emph{fit} values are small.
        However, consider the unit measure is \emph{exponential} (i.e. $10^{-1}, 10^{-2}$) the growth or decay rates can be significant.
    }
    \label{Experiment summary table}
\end{table}

\subsection{The Quantum Provider}
For this experiment, we are using the quantum emulator provided by Qiskit.
The QASM simulator is used to mimic an IBMQ device.
Additionally, the QASM simulator, by default, has no noise, so we can expect the result to be noise-free.
Note that the fault-free emulators do not reflect quantum devices precisely as the actual devices may suffer from various types of noise.
However, considering the allowed time span for these experiments, we will use the QASM simulator for T1-2022 and leave the actual quantum devices for future works.

\subsection{Creating Ansatzes}
We have chosen the \textit{NLocal} and \textit{TwoLocal} classes from the Qiskit circuit library to explore the two ansatzes structures

An example of circuits generated by Qiskit is visualised in Figure \ref{Ansatz samples}.
We can generate different ansatz by altering the repetition number and qubit number.
The circuit depth is the largest number of gate operations across all qubit registers in a circuit.
Furthermore, as the circuit high-level definition is translated into the gate set available on a given quantum machine, the circuit depth may significantly increase.
Obviously, as the ansatz repetition grows, the circuit depth also grows.
Figure \ref{Ansatz samples} further shows that the higher number of qubits leads to a deeper circuit for a fully entangled ansatz.

\begin{figure}
    \includegraphics[width=\textwidth]{Artefact/Appendices/ansatz3-2.png}
    \includegraphics[width=\textwidth]{Artefact/Appendices/ansatz3-3.png}
    \includegraphics[width=\textwidth]{Artefact/Appendices/ansatz4-2.png}
    \caption{
        Samples of parameterised circuits generated by the Qiskit framework with 'full entanglement' option.
        The ansatz is a sequence of rotation layers and entanglement layers.
        Above: an ansatz of three qubits and two repetition layers.
        Middle: an ansatz of three qubits and three repetition layers.
        Below: an ansatz of four qubits and two repetition layers.
    }
    \label{Ansatz samples}
\end{figure}

\subsection{Treatments for ansatzes}
Here we implement the two treatments for the ansatzes.
We select the first two methods from Section \ref{Research Design section} and applied them to the two selected ansatz.

\subsubsection{Method \#0: Unrestricted}
As discussed in the Section \ref{Research Design section}, the goal of this configuration is to produce a general multilayer perceptron network without any restriction.
The ansatzes will have unrestricted growth of circuit depth with a global cost function, the initial parameters of this ansatz is randomised.
We implement the default ansatz to have the number of qubits and repetition increased iteratively.

\subsubsection{Method \#1: Local Cost Function and Shallow Depth Implementation}
We implemented the \textit{Global Cost Function} as the measurement output for all qubits, while the \textit{Local Cost Function} is the measurement for the first two qubits.
Section \ref{Shallow Circuits, Local Cost Function section} and figure \ref{cost functions} previously explained the differences between the two cost functions.
The shallow ansatz is the same as compared with the default, however, the repetition number is kept as a constant number.

\subsection{Visualise the Variance} \label{Sec: Visualise the Variance}
We use the parameter shift rule from Eq. (\ref{Parameter-shift rules}) as implemented in the Qiskit Gradient library to calculate the gradient variance.
The BP phenomena can be verified when the gradient variance exponentially decreases with an increased number of qubits and repetition layers.

To visualise the gradient variances, we have plotted a range of random parameters for each ansatz as the initial starting point.
Such randomised parameters are generated 100 times uniformly to calculate the gradients.
We then plot the variance values of the gradients for different numbers of qubits and repetition values for a range of 2 to 7 qubits and ansatz layer repetition.
Note that the neural network generated in this experiment is not designed to answer a problem.
We will focus on the trainability of each method in the later phase of the experiment.

In short, we use 100 uniformly randomised parameters to scan the gradient.
Then we calculate the "slope" of the gradient.

\subsection{Classification Problem} \label{Sec: Classification Problem}
We implement a variational quantum classifier as previously discussed in the section \ref{VQA}.
The algorithm includes two stages: a training stage and a classification stage.
We use a dataset generated by Qiskit machine learning package for the algorithm.
The data was also used for a classification algorithm by Havlíček et al. \cite{havlicekSupervisedLearningQuantumenhanced2019}.

The quantum circuit for both stages is constructed from three parts: the feature map to encode data, the ansatz to perform optimisation and finally, a final rotation layer for measurement (see Figure \ref{Fig: Quantum circuit for classifier}).
The number of features will dictiate the number of qubits, in out case, we will be using two qubits for the quantum circuit.
For the method \#0 and \#2, there is no restriction on ansatz depth, so we build the ansatz to have 7 repetitions (15 depth units).
For the method \#1 with restriction for ansatz depth, we only use 2 reptitions (5 depth units).
For the customised ansatz in method \#3, we construct the ansatz to have the depth closer to that of method \#1 and \#2. We will be using three identity blocks as the ansatz, which results in 18 depth units.

The dataset consited of 50 labeled datapoints with two features is used for the traning stage.
The quantum circuit will be estimated as required by Gradient Descent algorithm provided by Qiskit to optimise the ansatz parameters.
For the classification satge, we use another set of 20 datapoints, and run the classifier with the optimised parameters obtained from the training stage.
The classifier would produce the predicted labels for each testing datapoint.
Then, we compare these predicted labels to the actual provided labels.

We collect the optimizer history (loss function per iteration) by implement a callback function for Gradient Descent algorithm.
The score $s$ of the classifier is calculate as the percentage likeliness of the predicted labels array compared to the actual labels array:
\begin{equation}
    s = 100\% (1 - \text{MSE})
\end{equation}
for $\text{MSE}$ is the mean square error of the actual labels array $Z$ and predicted labels array $\hat{Z}$ of length $n$:
\begin{equation}
    \text{MSE} = \frac{1}{n}\sum^n_{i=1}(Z_i - \hat{Z}_i)^2,
\end{equation}



\begin{figure}
    \centerline{
    \Qcircuit @C=1em @R=2em {
    \lstick{\ket{0}} & \multigate{1}{U_{\phi}(\vec{x})}    & \multigate{1}{W(\vec{\theta})}    & \meter & \rstick{z_1} \cw \\
    \lstick{\ket{0}} & \ghost{U_{\phi}(\vec{x})}           & \ghost{W(\vec{\theta})}           & \meter & \rstick{z_n} \cw \\
    }
    }
    \caption{
        Quantum Variational Classifier implemented for the experiment.
        The initial qubit state $\ket{0}^n$ is applied with a feature map $U_{\phi}(\vec{x})$ to encode data $\vec{x}$ from our dataset.
        Then, an unitary operation $W(\vec{\theta})$ is applied as the ansatz, follow up with a measurement layer.
        The output string $z \in \{0,1\}^n$ is mapped as label for the given datapoint $\vec{x}$.
    }
    \label{Fig: Quantum circuit for classifier}
\end{figure}


\subsection{Results and Analysis}

\begin{todolist}
\item mention a table that includes the number of qubits/layer/random parameter for each ansatz
\item describe more in figure \ref{Variance Local Cost}
\item analysis instead of describe
\item add the figures above (python notebook) for analysis
\item describe how alternating data leads to different result(s).
\end{todolist}

The section \ref{Development Process section} dicusses the configurations of the experiment. 
We summarize the experiment results as the table:
\begin{center}
    \begin{tabular}{|| c c c ||}
        \hline
        Ansatz      & Method                                & Variance of gradients \\[0.5ex] 
        \hline \hline
        NLocal      & None                                  & Decay                 \\
        \hline
        TwoLocal    & None                                  & Decay                 \\
        \hline
        NLocal      & Local Cost Function, Shallow circuit  & Sustain               \\
        \hline
        TwoLocal    & Local Cost Function, Shallow circuit  & Sustain               \\
        \hline
    \end{tabular}
\end{center}

For the default setting, the two ansatzes' gradient variances decay as expected.
As the number of qubits scales up, the variances decay exponentially, this indecates that the cost function landscape becomes flatter and flatter. 
We suspect that the result would be inefficientcy for any gradient-based optimization algorithm to train the model.
We discussed this phenomenon in Section \ref{Barren Plateaus section}.
Figure \ref{Plot ansatzes gradients default} shows the results of the two ansatzes in this configuration.

\begin{figure}
    \includegraphics[width=\textwidth]{Artefact/Appendices/NLocalDefault.png}
    \centerline{a) NLocal Ansatz gradient variance values}
    \includegraphics[width=\textwidth]{Artefact/Appendices/TwoLocalDefault.png}
    \centerline{b) TwoLocal Ansatz gradient variance values}
    \caption{
        The variances of gradient from differences ansatzes.
        On both plots: the variances vanish exponentially to the number of qubits, ansatzes are in default configuration.
    }
    \label{Plot ansatzes gradients default}
\end{figure}

In contrast, for the case of Local Cost Function and Shallow circuit, we observe that the variances of the ansatzes' gradient did not vanished when we attempt to increase the number of qubits.
This implies that the cost function landscape can sustain the slope.
Figure \ref{Variance Local Cost} shows the result of the experiment for Local Cost Function and Shallow circuit, in comparison with the default settings.

\begin{figure}
    \includegraphics[width=\textwidth]{Artefact/Appendices/variancesLCF.png}
    \caption{
        Comparison of the variance values of the two ansatzes with and without Local Cost Function and constant depth.
        The ansatzes with Global Cost Function and increased depth have their gradient variances decay exponentially with the number of qubits. 
    }
    \label{Variance Local Cost}
\end{figure}


\subsection{Artefact Development Summary}

We have implemented three methods of dealing with barren plateau by altering the ansatz's depth, cost function, and initial parameters aspects.
The experiments have produced the results as the slopes of the gradients for a number of qubits, as well as the performance of ansatzes in neural network training.
The results indicate that the variances of the gradient can be stable if we set a limit on the length of the circuit and the cost function, and the training performance can increase if we carefully select the initial parameters.

With this artefact, we have addressed the research design in Section \ref{Research Design section}.
We have implemented the three methods, namely \textit{local cost function, shallow ansatz depth}, \textit{layerwise learning} and \textit{identity blocks}.
We have compared the variances in the first phase of the experiment.
We anticipate that a higher variance value does not mean the optimisation is going in the right direction, the model can still stick it in a local minimum, or the error landscape is random.
To verify the performance of these methods, in the second phase, we implemented a variational quantum neural network to solve a classification problem with a standard dataset.

The experiments have verified that the unrestricted configuration may have run into a barren plateau, while the others can converge to their minimums - the answers.
Moreover, these experiments are conducted in an emulated environment with a noise model from \emph{ibm\_perth} backend.
Thus, this experiment can reflect the real-life situation to some extent.

According to Figure \ref{Fig: Plot Variances}, the variance values of limited depth - cost function method at seven qubits configuration is significantly higher than the rest.
At higher qubit configuration, the barren plateaus are also more likely to appear, and the performance of these methods under this phenomenon is worth investigating.
Due to time constrain, we leave these experiments for future works, including ansatzes of higher qubit count, and a dataset of higher dimension for classification.

